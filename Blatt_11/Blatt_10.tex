\documentclass[11pt,a4paper]{article}

\usepackage{amsmath} %for mathemathic formulas
\usepackage{amssymb}
\usepackage[ngerman]{babel} %for the german language by the spellling reform (without the package the date would look like April 20, 2020)
\usepackage{enumitem} %for enumeration surrounding 
\usepackage{graphicx} %for pictures
\usepackage{siunitx}
\usepackage{float}

\title{Blatt 10}
\date{\today}
\author{Hannah Rotgeri \and Feline Heinzelmann}

\begin{document}
    \maketitle

    \section*{Aufgabe 1}
	\begin{itemize}
		\item[a)] 
			Low-gain-FEL sind Oszillatoren wie bei einem konventionellen Laser, bei denen sich das Strahlungsfeld langsam innerhalb zweier Spiegel während vieler 
			Durchläufe von Elektronen durch den Undulator aufbaut. Die Elektronen werden dabei immer wieder in den Undulator eingespeist. 
			Die Verstärkung ist bei diesem FEL-Typ, wie der Name bereits erahnen lässt, sehr gering. 
            Das E-Feld $E_0$ lässt sich daher pro Umlauf als konstant nähern.
			Dadurch sind die Elektronen im Phasenraum auf sogenannten Phasenraumellipsen.
			Beim High-Gain-FEL erfolgt die gesamte Verstärkung in einem einizgen Durchlauf.
			Im Gegensatz dazu ist die Amplitude des E-Feldes $E_0$ beim high-gain-FEL ungleich 0 und ist abhängig von der Distanz $E_{x}(s)$, die das 
			Elektronenpaket im Undulator zurücklegt.
			Dies führt dazu, dass die Elektronen nach dem Ausbilden der Dichtemaximas wie eine Welle "überschwappen".
			Mikrobunching, welches bei beiden FEL-Typen auftritt, lässt sich als kollektives Phänomen bezeichnen.

		\item[b)]
			Viele unterschiedliche Frequenzen führen zu kürzeren Pulsen. 
			Realisiert werden, könnte dies durch einen unkonstanten Magnetabstand im Undulator.

	\end{itemize}


	
    \section*{Aufgabe 2}
	\begin{itemize}
		\item[a)] 
			Ein Elektronenpaket wird zwei Mal von sehr kurzen Laserpulsen moduliert.
			Diese sind so abgestimmt, dasss die Auslenkung der Elektronen im zentralen "Peak" maximal ist
			und die restlichen "Peaks" abgeschwächt werden.
			Dazu müssen die Laser unterschiedliche Wellenlängen haben.
			Nach dem durchlaufen der Chicane emittiern die stark ausglenkten Elektronen verstärkt Licht mit kurzer Gain-Länge aus.
		\item[c)]
			Das FWHM der Laserbandbreite darf nur 24\% sein, was an der Grenze des technisch Möglichen ist.
	\end{itemize}

	\section*{Aufgabe 3}

		Die Größenordung der Anstiegsrate scheint nicht zu passen. Theoretisch wäre der Strahl stabil, 
		wenn die Anstiegsrate geringer ist als die Dämpfung, die angegeben ist.
		Die angegebene Dämpfung entspricht aber einer Rate in der Größenordung $10³$.
		Die berechnete Anstiegsrate entspricht $10¹⁷$.
		Daher lässt sich auch in der b) nicht der gesuchte Radius bestimmen.
		Es zeigt sich, dass bei kleinerem Radius die Rate höher wird.
