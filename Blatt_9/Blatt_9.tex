\documentclass[11pt,a4paper]{article}

\usepackage{amsmath} %for mathemathic formulas
\usepackage{amssymb}
\usepackage[ngerman]{babel} %for the german language by the spellling reform (without the package the date would look like April 20, 2020)
\usepackage{enumitem} %for enumeration surrounding 
\usepackage{graphicx} %for pictures
\usepackage{siunitx}
\usepackage{float}

\title{Blatt 9}
\date{\today}
\author{Hannah Rotgeri \and Feline Heinzelmann}

\begin{document}
    \maketitle

    \section*{Aufgabe 1}
	\begin{itemize}
		\item[a)] 
			Der Wellenlängenbereich wird beim Seeding hauptsächlich durch die Wellenlänge des Seed-Lasers bestimmt. 
		\item[b)]
			Der Freie-Elektronen-Laser ist an sich kein Laser. Zwar wird beim FEL das Synchrotronlicht in seiner Intensität verstärkt, indem durch eine 
			externe Lichtquelle (Seeding) oder durch selbstverstärkte spontane Abstrahlung (SASE) die Elektronenpakete in Kleinstgruppen angeordnet werden
			und damit Strahlung der Elektronen in den Kleinstgruppen sich kohärent überlagert. Jedoch findet keine stimulierte Emission wie bei einem Laser statt.
			Das ausgesendete Licht eines FEL besitzt Lasereigenschaften, auch wenn es per Definition kein Laser ist.
	\end{itemize}


	
    \section*{Aufgabe 2}
	\begin{itemize}
		\item[a)] 
			\begin{equation*}
				L_{g} = \frac{1}{\sqrt{3}} (\frac{4 \gamma^3 m_{e}}{\mu_0 K^2 e^2 k_{u} n_{e}})^{1/3}
			\end{equation*}
			$L_{g} = \SI{0.375}{\meter}$
		\item[b)]
			\begin{align*}
				\gamma &= \frac{E_{e}}{E_{Ruhe}} = \frac{\SI{1.5e9}{\giga\electronvolt}}{\SI{511e3}{\giga\electronvolt} X}, \\
				K &= 2, \\
				\lambda_{U} &= \SI{25}{\centimeter} = \SI{0.25}{\meter}, \\
				n_{e} &= \frac{n}{V}, \\
				n &= \frac{Q}{e} = \frac{I \cdot U}{c \cdot e} = \frac{\SI{20}{\milliampere} \SI{115.2}{\meter}}{c \cdot e} = 4.8e10, \\
				V &= \sigma_{x} \cdot \sigma_{y} \cdot \sigma_{z}, \\
				\sigma_{x,y} &= \sqrt{\epsilon_{x,y} \cdot \beta}, \\
				\sigma_{x} &= \SI{1.265e-4}{\meter}, \\
				\sigma_{y} &= \SI{3.162e-5}{\meter}, \\
				\sigma_{z} &= c \cdot \sigma_{t} = \SI{0.03}{\meter}, \\
				n &= \frac{4.8*10^10}{  \SI{1.265e-4}{\meter} \cdot \SI{3.162e-5}{\meter} \cdot \SI{0.03}{\meter} } = \SI{4.0e18}{1/m^{-3}
			\end{align*}
			$L_{g} = \SI{11.1}{\meter}$
	\end{itemize}

	\section*{Aufgabe 3}
	Bei der Variation der Elektronendichte fällt auf, 
	dass bei höherer Dichte der Maximalwert früher erreicht wird.
	Das anfängliche E-Feld beeiflusst die höhe der Energieänderung.
	Es fällt auf, dass bei mehrfachem Durchlaufen des Programms, sich die Gainkurven trotz gleicher Parameter leicht unterscheiden.

\end{document}


