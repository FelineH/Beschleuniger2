\documentclass[11pt,a4paper]{article}

\usepackage{amsmath} %for mathemathic formulas
\usepackage{amssymb}
\usepackage[ngerman]{babel} %for the german language by the spellling reform (without the package the date would look like April 20, 2020)
\usepackage{enumitem} %for enumeration surrounding 
\usepackage{graphicx} %for pictures
\usepackage{siunitx}
\usepackage{float}

\title{Blatt 6}
\date{\today}
\author{Hannah Rotgeri \and Feline Heinzelmann}

\begin{document}
    \maketitle

    \section*{Aufgabe 1}
	\begin{itemize}
		\item[a)]
			Damit der Bunching-Faktor > 0 ist, muss es eine Struktur in der Ladungsverteilung geben, die kleiner ist als die Wellenlänge.
			Dabei ist das "Vorzeichen" dieser Strukur (also ob Peak oder Lücke) egal, da es nur auf die Fourier-Komponente ankommt.
		\item[b)]
			Durch die Dichtemodulation entsteht eine Art gezackte Schwingung (siehe Vorlesung Folie 96).
			Je stärker die Schikane, desto ausgeprägter sind die senkrechten Bereiche im Diagramm.
			Das Spektrum der kohärenten Strahlung besteht aus Peaks bei Vielfachen der Grundfrequenz.
	\end{itemize}


	
    \section*{Aufgabe 2}


		$\lambda = \frac{\lambda_u}{2\gamma²}(1+\frac{K²}{2})$

		$\Delta \lambda = \lambda_u \theta² /2$

		$\frac{\Delta \lambda}{\lambda} = \theta² \gamma² \frac{1}{1+K²/2} \approx \frac{1}{N}$

		$\to \theta² = \frac{1+K²/2}{N\gamma²}$

		mit $\theta = 0 \pm \sigma_{theta}  \to \sigma_{theta} = \frac{1}{\gamma} \sqrt{\frac{1+K²/2}{N}}$

		Wo das $\sqrt{1/2}$ herkommt, ist uns nicht klar.




\end{document}


