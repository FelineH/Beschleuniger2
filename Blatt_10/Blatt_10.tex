\documentclass[11pt,a4paper]{article}

\usepackage{amsmath} %for mathemathic formulas
\usepackage{amssymb}
\usepackage[ngerman]{babel} %for the german language by the spellling reform (without the package the date would look like April 20, 2020)
\usepackage{enumitem} %for enumeration surrounding 
\usepackage{graphicx} %for pictures
\usepackage{siunitx}
\usepackage{float}

\title{Blatt 10}
\date{\today}
\author{Hannah Rotgeri \and Feline Heinzelmann}

\begin{document}
    \maketitle

    \section*{Aufgabe 1}
	\begin{itemize}
		\item[a)] 
			Bei einem Low-Gain-FEL wird das Lichtfeld bei einmaligem durchlaufen des Undulators nicht genügend verstärkt.
			Daher muss das Lichtfeld die Struktur mehrfach durchlaufen. Das E-Feld lässt sich daher pro Umlauf als konstant nähern.
			Dadurch sind die Elektronen im Phasenraum auf sogenannten Phasenraumelipsen.
			Beim High-Gein-FEL reicht die Verstärkung des Lichtfelds bei einem Umalauf aus.
			Das E-Feld ist somit nichtmehr konstant.
			Dies führt dazu, dass die Elektronen nach dem Ausbilden der Dichtemaximas wie eine Welle "überschwappen".
			Mikrobunching, welches bei beiden FEL-Typen auftritt, lässt sich als kollektives Phänomen bezeichnen.
		\item[b)]
			Viele unterschiedliche Frequenzen führen zu kürzeren Pulsen. 
			Realisiert werden, könnte dies durch einen unkonstanten Magnetabstand im Undulator.

	\end{itemize}


	
    \section*{Aufgabe 2}
	\begin{itemize}
		\item[a)] 
			Ein Elektronenpaket wird zwei Mal von sehr kurzen Laserpulsen moduliert.
			Diese sind so abgestimmt, dasss die Auslenkung der Elektronen im zentralen "Peak" maximal ist
			und die restlichen "Peaks" abgeschwächt werden.
			Dazu müssen die Laser unterschiedliche Wellenlängen haben.
			Nach dem durchlaufen der Chicane emittiern die stark ausglenkten Elektronen verstärkt Licht mit kurzer Gain-Länge aus.
		\item[c)]
			Das FWHM der Laserbandbreite darf nur 24\% sein, was an der Grenze des technisch Möglichen ist.
	\end{itemize}

	\section*{Aufgabe 3}


