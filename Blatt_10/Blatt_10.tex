\documentclass[11pt,a4paper]{article}

\usepackage{amsmath} %for mathemathic formulas
\usepackage{amssymb}
\usepackage[ngerman]{babel} %for the german language by the spellling reform (without the package the date would look like April 20, 2020)
\usepackage{enumitem} %for enumeration surrounding 
\usepackage{graphicx} %for pictures
\usepackage{siunitx}
\usepackage{float}

\title{Blatt 9}
\date{\today}
\author{Hannah Rotgeri \and Feline Heinzelmann}

\begin{document}
    \maketitle

    \section*{Aufgabe 1}
	\begin{itemize}
		\item[a)] 
			Low-gain-FEL sind Oszillatoren wie bei einem konventionellen Laser, bei denen sich das Strahlungsfeld langsam innerhalb zweier Spiegel während vieler 
			Durchläufe von Elektronen durch den Undulator aufbaut. Die Elektronen werden dabei immer wieder in den Undulator eingespeist. 
			Die Verstärkung ist bei diesem FEL-Typ, wie der Name bereits erahnen lässt, sehr gering.
			Die Amplitude des elektrischen Feldesm $E_0$ wird als nahezu konstant angenommen. \newline
			Im Gegensatz dazu ist die Amplitude des E-Feldes $E_0$ beim high-gain-FEL ungleich 0 und ist abhängig von der Distanz $E_{x}(s)$, die das 
			Elektronenpaket im Undulator zurücklegt.
			Denn die gesamte Verstärkung erfolgt in einem einizgen Durchlauf.
		\item[b)]
			

	\end{itemize}


	
    \section*{Aufgabe 2}
	\begin{itemize}
		\item[a)] 
			\begin{equation*}
				L_{g} = \frac{1}{\sqrt{3}} (\frac{4 \gamma^3 m_{e}}{\mu_0 K^2 e^2 k_{u} n_{e}})^{1/3}
			\end{equation*}
			$L_{g} = \SI{0.375}{\meter}$
		\item[b)]
			\begin{align*}
				\gamma &= \frac{ E_{e} }{ E_{Ruhe} } = \frac{ \SI{1.5e9}{\gigaelectronvolt} }{ \SI{511e3}{\gigaelectronvolt} }, \\
				K &= 2, \\
				\lambda_{U} &= \SI{25}{\centimetre} = \SI{0.25}{\metre}, \\
				n_{e} &= \frac{n}{V}, \\
				n &= \frac{Q}{e} = \frac{I \cdot U}{c \cdot e} = \frac{ \SI{20}{\milliampere} \SI{115.2}{\metre}}{c \cdot e} = 4.8*10^{10}, \\
				V &= \sigma_{x} \cdot \sigma_{y} \cdot \sigma_{z}, \\
				\sigma_{x,y} &= \sqrt{\epsilon_{x,y} \cdot \beta}, \\
				\sigma_{x} &= \SI{1.265e-4}{\metre}, \\
				\sigma_{y} &= \SI{3.162e-5}{\metre}, \\
				\sigma_{z} &= c \cdot \sigma_{t} = \SI{0.03}{\metre}, \\
				n &= \frac{4.8*10^10}{  \SI{1.265e-4}{\metre} \cdot \SI{3.162e-5}{\meter} \cdot \SI{0.03}{\metre} } = 4.0*10^{20} \, 1/m^{3}
			\end{align*}
			$L_{g} = \SI{2.45}{\metre}$
	\end{itemize}

	\section*{Aufgabe 3}
	
	Die Bilder ähneln den Bildern im Skript S.47. Die Phasenraumverteilung der Elektronen am Anfang des Undulators $\SI{0}{\metre}$ zeigt Bild \ref{fig:0meter}. Mit zunehmeneder Wegstrecke von $\SI{0}{\metre}$ 
	bis $\SI{6}{\metre}$ setzt Verstärkungsprozess ein und es bilden sich Wirbel aus, die die Sättigung anzeigen \ref{fig6meter}.
	Die Gain-Kurve kann durch höheres E-Feld nach vorne verschoben werden.

	\begin{figure}
		\includegraphics[width=\textwidth]{build/Blatt9_Aufgabe3_s1_Feline,Hannah.pdf}
		\centering
		\label{fig:0meter}
	\end{figure}

	\begin{figure}
		\includegraphics[width=\textwidth]{build/Blatt9_Aufgabe3_s4_Feline,Hannah.pdf}
		\centering
		\label{fig:4meter}
	\end{figure}

	\begin{figure}
		\includegraphics[width=\textwidth]{build/Blatt9_Aufgabe3_s6_Feline,Hannah.pdf}
		\centering
		\label{fig:6meter}
	\end{figure}

 	\begin{figure}
		\includegraphics[width=\textwidth]{build/Leistung,E_0=(1000000+0j),n_e=1e+22.pdf}
		\centering
		\caption{Zunächst zeigt sich ein exponentieller Anstieg, der in eine Sättigung übergeht. Nach dem Verstärkungsmaximum würde man theoretisch eine Oszillation beobachten mit weiter zunehmender Wegstrecke}
		\label{fig:gain}
	\end{figure}

	Bei der Variation der Elektronendichte fällt auf, 
	dass bei höherer Dichte der Maximalwert früher erreicht wird.
	Das anfängliche E-Feld beeiflusst die höhe der Energieänderung.
	Es fällt auf, dass bei mehrfachem Durchlaufen des Programms, sich die Gainkurven trotz gleicher Parameter leicht unterscheiden.

\end{document}


