\documentclass[11pt,a4paper]{article}

\usepackage{amsmath} %for mathemathic formulas
\usepackage{amssymb}
\usepackage[ngerman]{babel} %for the german language by the spellling reform (without the package the date would look like April 20, 2020)
\usepackage{enumitem} %for enumeration surrounding 
\usepackage{graphicx} %for pictures
\usepackage{siunitx}
\usepackage{float}

\title{Blatt 7}
\date{\today}
\author{Hannah Rotgeri \and Feline Heinzelmann}

\begin{document}
    \maketitle

    \section*{Aufgabe 1}
	\begin{itemize}
		\item[a)]
			Die ponderomotorische Phase $\Phi_{+}$ bezieht sich auf den mittleren konstanten Energieübertrag beim kontinuierlichen Energieaustausch zwischen Elektron und Lichtfeld, 
            der im inhomogenen Feld innerhalb des Freien-Elektronen-Lasers geschieht. 
            Die pondomotorische Phase muss im zeilichen Mittel also null sein (Bedingung: $\lambda = 2 \pi /k$). 
		\item[b)]
			Die andere Phase $\Phi_{-} = \Phi_{+} - 2k_{u}s$ ist von $\Phi_{+}$ abhängig und oszilliert. Deswegen ändert sich auch die Elektronenenergie.
            Die pondomotorische Phase ist von der Energie abhängig. Grund dafür ist, wenn der Lorenzfaktor von Resonanzbedingung abweicht, weil das Elektron dann nicht mehr konstnat um eine Wellenlänge des Lasers pro Periode zurückbleibt.
	\end{itemize}


	
    \section*{Aufgabe 2}
		Siehe beigefügtes Pdf.

	\section*{Aufgabe 3}

        \begin{figure}
			\centering
			\includegraphics[width=\textwidth]{images/CHG.png}
            \caption{Aufbau eines CHG (coherent harmonic generation) mit Modulation der Elektronenenergie durch einen kurzen Laserpuls in einem Undulator 
            (Modulator) und Dichtemodulation im nachfolgenden Undulator (Radiator) für kohärente Abstrahlung des kurzen Pulses}
		\end{figure}


		\begin{figure}
			\centering
			\includegraphics[width=\textwidth]{build/a.pdf}
		\end{figure}

		\begin{figure}
			\centering
			\includegraphics[width=\textwidth]{build/b.pdf}
		\end{figure}

		\begin{figure}
			\centering
			\includegraphics[width=\textwidth]{build/cd.pdf}
		\end{figure}

		\begin{figure}
			\centering
			\includegraphics[width=\textwidth]{build/ef.pdf}
		\end{figure}

\end{document}


