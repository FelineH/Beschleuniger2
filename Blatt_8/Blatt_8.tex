\documentclass[11pt,a4paper]{article}

\usepackage{amsmath} %for mathemathic formulas
\usepackage{amssymb}
\usepackage[ngerman]{babel} %for the german language by the spellling reform (without the package the date would look like April 20, 2020)
\usepackage{enumitem} %for enumeration surrounding 
\usepackage{graphicx} %for pictures
\usepackage{siunitx}
\usepackage{float}

\title{Blatt 8}
\date{\today}
\author{Hannah Rotgeri \and Feline Heinzelmann}

\begin{document}
    \maketitle

    \section*{Aufgabe 1}
	\begin{itemize}
		\item[a)] 
			Magnetfeld der elmagn Welle 90T. 
			Kraft hebt sich mit Querkraft des E-Felds auf.
			
		\item[b)]
			Bei der laserinduzierten Energiemodulation, findet die Wechselwirkung nur statt,
			wenn die Elektronen im Laserpuls sind. 
			Da die Elektronen pro Undulatorperiode einen optischen Zyklus nach hinten rutschen,
			sind die Elektronen nach dieser Anzahl an Undulatorperioden nicht mehr im Laserpuls.
			Bei einem high-gain FEL findet die Wechselwirkung zwischen Elektronen und dem emittierten Licht dieser statt,
			daher befinden sich die Elektronen immer im "Lichtpuls".

	\end{itemize}


	
    \section*{Aufgabe 2}
	\begin{itemize}
		\item[a)] 
			In einem Speicherring werden hochenergetische Elektronen mit einem niederenergetischen Elektronenstrahl (aus Linearbeschleuniger) beschossen.
			Dabei werden kurze Elektronenpakete "rausgeschossen".
			Diese strahlen ultrakurze Puls ab.
			Verglichen mit dem Femtoslicing (im Paper: laser slicing) ist der Strom pro Puls wesentlich höher (Faktor 6 bis 10),
			die Wiederholungsrate ist höher und die Pulslänge ist ähnlich.
			
		\item[b)]
			Höhere Energien und Ladung des Linearbeschleuniger Elektronenpakets führt zu einem größeren Kickwinkel 
			und verbessert die Performance bei harter Röntgenstrahlung.
			Verringerung des Eintreffwinkels, kan die Pulslänge verringern, aber erhöht den Untergrund.
			Zur besseren Synchronisation könnte ein Laseroptisches System verwendet werden.

		\item[c)]
			Strahlenergie 1,5 GeV $(\gamma  = 3000)$, Beta-Funktionen $\beta_{x,y}
			= 5\,m$, vertikale Emittanz $\epsilon_y = 10^{-10}\,m\,rad$, Standardabweichung der Paketlänge $\sigma_z = 43\,ps$. \\
			$\sigma_{x,y} = \sqrt{\epsilon_{x,y} \beta_ {x,y}} = \SI{2.24e-5}{\meter}\\
			\sigma_{z} = \SI{0.013}{\meter} $(mit c multipliziert)$\\
			\rho = \sqrt{\frac{\gamma_2 ² (\sigma_x² + \sigma_y²)}{(\gamma_2²+1)\sigma_y²}} = 0,003.\\
			$(Es ist aber nur $\gamma_{1}$ gegeben und ich habe keine Ahnung, wie man das ausrechnen soll, daher haben wir das aus dem Paper genommmen)
	\end{itemize}

	\section*{Aufgabe 3}
		Die Ergebnisse entsprechen den Erwartungen (vgl mit Vorlesung).

\end{document}


