\documentclass[11pt,a4paper]{article}

\usepackage{amsmath} %for mathemathic formulas
\usepackage{amssymb}
\usepackage[ngerman]{babel} %for the german language by the spellling reform (without the package the date would look like April 20, 2020)
\usepackage{enumitem} %for enumeration surrounding 
\usepackage{graphicx} %for pictures
\usepackage{siunitx}
\usepackage{float}

\title{Blatt 8}
\date{\today}
\author{Hannah Rotgeri \and Feline Heinzelmann}

\begin{document}
    \maketitle

    \section*{Aufgabe 1}
	\begin{itemize}
		\item[a)]
			Die ponderomotorische Phase $\Phi_{+}$ bezieht sich auf den mittleren konstanten Energieübertrag beim kontinuierlichen Energieaustausch zwischen Elektron und Lichtfeld, 
            der im inhomogenen Feld innerhalb des Freien-Elektronen-Lasers geschieht. 
            Die pondomotorische Phase muss im zeilichen Mittel also null sein (Bedingung: $\lambda = 2 \pi /k$). 
		\item[b)]
			Die andere Phase $\Phi_{-} = \Phi_{+} - 2k_{u}s$ ist von $\Phi_{+}$ abhängig und oszilliert. Deswegen ändert sich auch die Elektronenenergie.
            Die pondomotorische Phase ist von der Energie abhängig. Grund dafür ist, wenn der Lorenzfaktor von Resonanzbedingung abweicht, weil das Elektron dann nicht mehr konstnat um eine Wellenlänge des Lasers pro Periode zurückbleibt.
	\end{itemize}


	
    \section*{Aufgabe 2}
	\begin{itemize}
		\item[a)] 
			In einem Speicherring werden hochenergetische Elektronen mit einem niederenergetischen Elektronenstrahl (aus Linearbeschleuniger) beschossen.
			Dabei werden kurze Elektronenpakete "rausgeschossen".
			Diese strahlen ultrakurze Puls ab.
			Verglichen mit dem Femtoslicing (im Paper: laser slicing) ist der Strom pro Puls wesentlich höher (Faktor 6 bis 10),
			die Wiederholungsrate ist höher und die Pulslänge ist ähnlich.
			
		\item[b)]

		\item[c)]
			Strahlenergie 1,5 GeV $(\gamma  = 3000)$, Beta-Funktionen $\beta_{x,y}
			= 5\,m$, vertikale Emittanz $\epsilon_y = 10^{-10}\,m\,rad$, Standardabweichung der Paketlänge $\sigma_z = 43\,ps$. \\
			$\sigma_{x,y} = \sqrt{\epsilon_{x,y} \beta_ {x,y}} = \SI{2.24e-5}{\meter}\\
			\sigma_{z} = \SI{0.013}{\meter} $(mit c multipliziert)$\\
			\rho = \sqrt{\frac{\gamma_2 ² (\sigma_x² + \sigma_y²)}{(\gamma_2²+1)\sigma_y²}} = 0,003.\\
			$(Es ist aber nur $\gamma_{1}$ gegeben und ich habe keine Ahnung, wie man das ausrechnen soll, daher haben wir das aus dem Paper genommmen)
	\end{itemize}

	\section*{Aufgabe 3}


\end{document}


