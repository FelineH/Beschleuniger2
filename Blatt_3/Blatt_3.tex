\documentclass[11pt,a4paper]{article}

\usepackage{amsmath} %for mathemathic formulas
\usepackage{amssymb}
\usepackage[ngerman]{babel} %for the german language by the spellling reform (without the package the date would look like April 20, 2020)
\usepackage{enumitem} %for enumeration surrounding 
\usepackage{graphicx} %for pictures
\usepackage{siunitx}

\title{Blatt 3}
\date{\today}
\author{Hannah Rotgeri \and Feline Heinzelmann}

\begin{document}
    \maketitle

    \section*{Aufgabe 1}

    \begin{enumerate}
        \item[a)] Als Kenngröße für das Spektrum der Synchrotronstrahlung wird die kritische Frequenz $\omega_c = \frac{3c\gamma³}{2R}$ verwendet.
         \begin{itemize}
             \item[i)] Die Energie ist proportional zu $\gamma$. Da dieser in der dritten Potenz eingeht, verachtfacht sich die kritische Frequenz.
             \item[ii)] Die Magnetfeldstärke ist antiproportional zum Radius und damit proportional zu der kritischen Frequenz. Demnach verdoppelt diese sich.
             \item[ii)] Der Radius ist proportional zum sich verdoppelnden Impuls und antiproportional zur sich verdoppelnden Ladung. Damit ändert sich die Lage des Spektrums nicht.
         \end{itemize}    
        \item[b)] Durch Überlagerung der Strahlung kommt es zu Interferrenzerscheinungen (eine Art Schwebung).
		
	\end{enumerate}
	
    \section*{Aufgabe 2}
	
        siehe Übungsblatt3,A2,HannahR,FelineH.pdf
        

    \section*{Aufgabe 3}

    \begin{figure}[h]
        \centering
        \includegraphics[width=0.7\textwidth]{build/spektrum_synchrotronstrahlung.pdf}
    \end{figure}

    \begin{figure}[h]
        \centering
        \includegraphics[width=0.7\textwidth]{build/spektrum_synchrotronstrahlung_logarithmisch.pdf}
    \end{figure}

    \begin{enumerate}
        \item[a)] Die erstellten Plots zeigen das Spektrum der Synchrotronstrahlung.
        \item[b)] Der empfindliche Bereich der Diode ist in den Plotts als rosa Bereich eingezeichnet.
            In der nicht-logarithmischen Darstellung ist dieser nur als Gerade zu erkennen.
            In der logarithmischen Darstellung ist erkennbar, 
            dass die Diode lediglich das untere Ende des Spektrums detektiert.
            Um den Andeil der detektierten Leisung zu bestimmen,
            wird zunächst der Anteil des Spektrums berechnet für den die Diode empfindlich ist.
            Daraufhin wird analog zu der Aufgabe 2 von Blatt 2 der Raumwinkelanteil berechnet den die Diode abdeckt.
            Damit ergibt sich ein Anteil von $\num{4.81e-8}$ an der gesamten abgestrahlten Leistung.  
    

    \end{enumerate}



\end{document}


