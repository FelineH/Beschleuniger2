\documentclass[11pt,a4paper]{article}

\usepackage{amsmath} %for mathemathic formulas
\usepackage{amssymb}
\usepackage[ngerman]{babel} %for the german language by the spellling reform (without the package the date would look like April 20, 2020)
\usepackage{enumitem} %for enumeration surrounding 
\usepackage{graphicx} %for pictures
\usepackage{siunitx}

\title{Blatt 4}
\date{\today}
\author{Hannah Rotgeri \and Feline Heinzelmann}

\begin{document}
    \maketitle

    \section*{Aufgabe 1}
	\begin{itemize}
		\item[a)]
			Wenn Elektronen relativistisch beschleunigt werden in einer Vakuumkammer, stauchen sich die E-Feld-Linien longitudinal zusammen. 
			Bei Änderung des Kammerquerschnitts werden die Spiegelladungen transersal beschleunigt. Durch diesen Prozess ensteht Synchrotronstrahlung im entferntesten Sinne, da eine Art Hertzscher Dipol gebildet wird.
			Grundsetzlich beschreiben Spiegelladungen hier, auch wenn es sich dabei nicht um reale Teilchen handelt, das sich ändernde Feld des realen Teilchens.
		\item[b)]
			Normalerweise würde man erwarten, dasss die Zeit, die mindestens vergehen muss, damit der Detektor anspricht, sich durch
			$t_{min} = \frac{s}{v} = \frac{\num{30e-3}}{c} = \SI{1e-10}{\second}$ errechnen lässt. Jedoch wird die ausgesendete em Welle instantan am Detektor 
			gemessen, da sich das Feld mit der Ladung bewegt.
	\end{itemize}


	
    \section*{Aufgabe 2}

	\begin{itemize}
		\item[a)]
        	$\frac{\symup{d}}{\symup{d}t}\biggl(\frac{\vec{n}\times(\vec{n}\times\vec{\beta})}{1-\vec{\beta}\vec{n}}\biggr) &\stackrel{\stackrel{\text{Graßmann-}}{\text{Identität}}}{=} \frac{\symup{d}}{\symup{d}t}\biggl(\frac{\vec{n}(\vec{n}\vec{\beta})-\vec{\beta}}{1-\vec{\beta}\vec{n}}\biggr) \\
			&\stackrel{\text{Produktregel}}{=} \frac{\vec{n}(\vec{n}\dot{\vec{\beta}})(1-\vec{\beta}\vec{n})+(\dot{\vec{\beta}}\vec{n})\vec{n}(\vec{n}\vec{\beta})-(\dot{\vec{\beta}}(1-\vec{\beta}\vec{n})+\vec{\beta}(\dot{\vec{\beta}}\vec{n}))}{(1-\vec{\beta}\vec{n})^2} \\
			&= \frac{\vec{n}(\vec{n}\dot{\vec{\beta}})-\cancel{\vec{n}(\vec{n}\dot{\vec{\beta}})(\vec{\beta}\vec{n})}+\cancel{(\dot{\vec{\beta}}\vec{n})\vec{n}(\vec{n}\vec{\beta})}-\dot{\vec{\beta}}+\dot{\vec{\beta}}(\vec{\beta}\vec{n})-\vec{\beta}(\dot{\vec{\beta}}\vec{n})}{(1-\vec{\beta}\vec{n})^2} \\
			&= \frac{(\vec{n}\dot{\vec{\beta}})(\vec{n}-\vec{\beta})-\dot{\vec{\beta}}(\vec{n}(\vec{n}-\vec{\beta}))}{(1-\vec{\beta}\vec{n})^2} \\
			&\stackrel{\stackrel{\text{Graßmann-}}{\text{Identität}}}{=} \frac{\vec{n}\times((\vec{n}-\vec{\beta})\times\dot{\vec{\beta}})}{(1-\vec{\beta}\vec{n})^2} \quad \checkmark$
        \item[b)]
			Strahlungsfeld des elektrischen Feldes einer Ladung $e$:
			\begin{align*}
				\vec{E}(t) &= \frac{e}{4 \pi \epsilon_{0} c} \cdot \frac{\vec{n} \times \{(\vec{n} - \beta) \times \dot{\vec{\beta}} \} }{r (1-\vec{n} \vec{\beta})^3} \bigg \vert_{ret} \\
							&= \frac{e}{4 \pi \epsilon_{0} c} \cdot \frac{ \begin{pmatrix} \sin{\theta} \\ 0 \\ \cos{\theta} \end{pmatrix} \times \Biggl\{ \Biggl(\begin{pmatrix} \sin{\theta} \\ 0 \\ \cos{\theta} \end{pmatrix} - \begin{pmatrix} 0 \\ 0 \\ \beta \end{pmatrix} \Biggr) \times \begin{pmatrix} \beta^2 c/R \\ 0 \\ 0 \end{pmatrix} \Biggr\} }{r \Biggl( 1-\begin{pmatrix} \sin{\theta} \\ 0 \\ \cos{\theta} \end{pmatrix} \cdot \begin{pmatrix} 0 \\ 0 \\ \beta \end{pmatrix} \Biggr)^3 } \\
							&= \frac{e}{4 \pi \epsilon_{0} c} \cdot \frac{ \Biggl( \begin{pmatrix} -\cos{\theta} (\cos{\theta - \beta}) \beta^2c/R \\ 0 \\ \sin{\theta}(\cos{\theta} - \beta) \beta^2c/R \end{pmatrix} \Biggr) }{r (1 - \beta \cos{\theta})^3} \, \text{mit} \, r = R(1+\beta \cos{\theta}) \\
			\end{align*}
	\end{itemize}
    \section*{Aufgabe 3}
		\begin{itemize}
			\item[a)] Zeichne E-Feld-Vektoren in einem Abstand $a=1$ vom Ursprung (Punkt P des Teilchens zum aktuellen Zeitpunkt) als Funktion des Winkels in 1 Grad Schritten:
			\begin{figure}[ht]
				\centering
				\includegraphics[width=0.7\textwidth]{build/E_beta1.pdf}
			\end{figure}

			\begin{figure}[ht]
				\centering
				\includegraphics[width=0.7\textwidth]{build/E_beta50.pdf}
			\end{figure}

			\begin{figure}[ht]
				\centering
				\includegraphics[width=0.7\textwidth]{build/E_beta99.pdf}
			\end{figure}

			\item[b)] Stelle in einem Polardiagramm den Betrag des elektrischen Feldes der bewegten Ladung geteilt durch den Betrag des Feldes 
			einer am Punkt P ruhenden Ladung ($\beta=0$) als Funktion des Winkels in 1 Grad Schritten dar:
			\begin{figure}[ht]
				\centering
				\includegraphics[width=0.7\textwidth]{build/field_ratio_beta50.png}
			\end{figure}

			\begin{figure}[ht]
				\centering
				\includegraphics[width=0.7\textwidth]{build/field_ratio_beta99.png}
			\end{figure}

		\end{itemize}







\end{document}


